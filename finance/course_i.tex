\section{Foundations of Modern Finance I}
\subsection*{Present value of annuities and perpetuities}
Perpetuity: $\dfrac{CF}{r}$. \\ Growing perpetuity: $\dfrac{CF}{r -g}$. \\ Annuity: $\dfrac{CF}{r}\cdot\left(1-\dfrac{1}{(1+r)^n}\right)$. \\ Annuity with growth: \\ $\dfrac{CF}{r-g}\cdot\left(1 - \dfrac{(1+g)^n}{(1+r)^n}\right)$, \\where $r$ rate of return, $g$ growth rate, $n$ compounding periods
\subsection*{Arrow-Debrew securities. State-space model}
$\phi_1\dots\phi_n$ state prices \\
$p_1\dots p_n$ state probabilities, \\ where $\sum p_i = 1$ \\
$X_1\dots X_n$ state payouts \\
$P = \sum\phi_i\cdot X_i  = \dfrac{E(P)}{(1+\bar{r})}, \\ \bar{r} = \dfrac{E(P) - P}{P} = \dfrac{\sum p_i \cdot X_i}{P} - 1, \\ E(P) = \sum p_i \cdot X_i = P\cdot(1+\bar{r})$, \\ where $P$ is price, $E(P)$ is expected payout, $\bar{r}$ is expected return.
\subsection*{Discounted cash flow and rate of return}
$r$ is rate of return, $r_f$ is risk-free rate of return, $r - r_f$ is excess return \\
$ r = \dfrac{D_1 + P_1 - P_0}{P_0} = \dfrac{D_1 + P_1}{P_0} - 1, \\ P_0 = \dfrac{D_1+P_1}{1+r}$ \\
With $g$ as growth rate, \\ $P=\dfrac{D}{r-g}, g=\dfrac{D_1}{D_0} - 1$ \\
$\bar{r}=E(r)$, expected return, $\pi = \bar{r} - r_f$ is risk premium
\subsection*{Relation between real and nominal cash flows}
$r_{real}=\dfrac{1+r_{nominal}}{1+inflation} - 1$ \\
For nominal flow,\\ $CF\cdot(1+r_{real})\cdot(1+inflation)$ \\
For real flow, \\ $CF\cdot(1+r_{real})$
\subsection*{Accounting}
$I_t = EPS_t\times b$, where $b$ - plowback rate\\
$EPS_{t+1} = EPS_t+I_t\times ROI_t$ \\
$BVPS_{t+1}=BVPS_t + I_{t+1}$ \\
$D_t=EPS_t\times (1-b_t)$ \\

Growth rate $g = \dfrac{EPS_{t+1}}{EPS_t} - 1$ \\
With growth $P_0 = \dfrac{D}{r-g}$\\
Without growth $P^{nogrowth}_0 = \dfrac{D}{r}$, where $g = 0$ and $b = 0$ \\
Growth opportunity $PVGO = P_0 - P^{nogrowth}_0$\\
Horizon value estimation: \\
$PV(Free cash flow) +$ $P/E$ ratio or $P/B$ ratio or $DCF$
\subsection*{Risk}
Expected utility \\
$E[u(x)] = \sum p_i\cdot u(P_i)$, \\ where $p_i$ is probability, $P_i$ is payout \\
Expected payoff \\
$E(P) = \sum p_i\cdot P_i$ \\
Relative risk aversion \\
$RRA(W) = -\dfrac{W\cdot u''(W)}{u'(W)}$ \\
Certainty equivalent $CE = u^{-1} (E(u(x)))$ \\
$\pi$ - sure loss, risk premium, $W$ is investment amount, so \\
$E(u(W\cdot (1+x))) = u(W\cdot (1-\pi))$\\
or \\
$CE = W\cdot (1-\pi), \pi =1 - \dfrac{CE}{W}$ \\

$$
\begin{cases}
  +x\%, p_1 \\
  -x\%, p_2
\end{cases}
$$ \\
$E(u(W\cdot (1+x))) = \sum p_i \cdot u(W\cdot (1+x))$
\subsection*{Interest rate conversion EAR/APR}
$T$ - compounding interval, fraction \\
yearly: $T = 1$ \\
monthly: $T = \frac{1}{12}$ \\
daily: $T = \frac{1}{365}$ \\
$P$ - principal
$n$ - number of payment periods, so period payment $M$ is \\
$M = P\cdot \dfrac{APR\cdot (1+APR\cdot T)^n}{(1+APR\cdot T)^n - 1}$ \\
$\lim_{T\to 0} 1 + EAR = e^{APR}$, so \\ $APR = \ln(1+EAR)$ \\
$APR = \dfrac{(1+EAR)^T - 1}{T}$ \\
$1+EAR = (1+T\cdot APR)^{\frac{1}{T}}$ \\
\subsection*{Duration}
Discount bond price $B_t = (1+y)^{-t}$, discounted bond duration is $t$, so modified duration is $MD = \dfrac{t}{1+y}$ \\
Macaulay duration is \\ $D = \dfrac{1}{B}\cdot \sum_t \dfrac{CF_t}{(1+y)^t} \cdot t$\\
Modified duration is $MD = \dfrac{D}{1+y}$\\
Modified duration for perpetuity is $MD = \dfrac{1}{y}$, so Macaulay duration is $D=MD\cdot (1+y) = \dfrac{1+y}{y}$
\subsection*{Duration based approximations}
$\Delta{y}$ is the change in the interest rate, $P$ is the asset price. \\
$\Delta{P} = -P\times MD\times \Delta{y}$ \\
Convexity $CX$ is \\
$CX = \dfrac{1}{2}\cdot \dfrac{1}{P}\cdot \dfrac{1}{(1+y)^2} \cdot \sum_{t} PV(CF_t) \cdot t \cdot (t+1)$, convexity based approximation is\\
$\Delta{P} = P\times (-MD \cdot \Delta{y} + CX\cdot \Delta{y}^2)$
\subsection*{Statistic}
Excel functions: \\
Sample mean $AVG()$ \\ 
Standard deviation $STDEV.S()$ \\
Covariance: \\ $cov = \dfrac{1}{T-1}\cdot \sum (r_A - \bar{r}_A)\cdot(r_B-\bar{r}_B)$ \\
Corellation: $corr = \dfrac{cov}{SD(A)\cdot SD(B)}$\\
Portfolio variance: \\
$cov_{ij} = SD_i\cdot SD_j \cdot corr_{ij}$\\
$Var[P_{AB}] = \sum w_i^2 \cdot SD_i^2 + \sum_{i\neq j} 2w_iw_j \cdot SD_iSD_j \cdot corr_{ij}$, \\
$Var[P] = \dfrac{1}{n}\cdot SD^2 + \left(1-\dfrac{1}{n}\right)\cdot corr\cdot SD \cdot SD$, \\
where $SD$ is an average standard deviation
\subsection*{APT}
For well diversified portfolios: \\
$\widetilde{r}_P = \bar{r}_P + \sum b_i\cdot f_i$, where $\bar{r}_P$ is expected return\\
$\bar{r}_P - r_f = \lambda\cdot\beta_P$, where $r_f$ is risk free rate, $\lambda$ is risk price and $\beta_P$ is factor loading for single factor portfolio.
Same $\bar{r}_P - r_f = \sum_i \lambda_i\cdot\beta_i$ for $i$ factors portfolio\\
Return variance:\\
$Var(r) = \sum_i \beta_i^2\cdot Var(f_i) + Var(\epsilon)$\\
Covariance: \\
$cov(A,B) = \sum_i\beta_{i,A}\beta_{i,B}\cdot Var(f_i)$
\subsection*{APT in Excel}
$r_i - r_f = \alpha + \beta_1(r_1-r_f) + \beta_2(r_2 - r_f) + \epsilon_i$
To estimate $\beta_1$, $\beta_2$ and $\alpha$ (in this order): \\
$=LINEST(\alpha, \beta_2, \beta_1)$ (reverse order)
\subsection*{Capital investment}
$CF = OpRev - OpEx - Tax - CapEx$\\
$OpProfit = OpRev - OpEx$ \\
$Tax = \tau\cdot OpProfit - \tau\cdot Depreciation$ \\
$CF = (1-\tau)\cdot OpProfit - CapEx + \tau \cdot Depreciation$ \\
Work capital:\\
$WC = Inventory + A/R - A/P$, where $A/R$ accounts receivable, $A/P$ accounts payable\\
$CF = (1-\tau)\cdot OpProfit + \tau\cdot Depreciation - CapEx - \Delta WC$
\subsection*{Alternatives to NPV}
Payback period\\
Choose $S$ so $PB = S$, $\sum_{i=1}^S CF_i \geq - CF_0$ 
Discounted payback period: \\
$DPB = S$, $\sum_{i=1}^S \dfrac{CF_i}{(1+r)^i} \geq - CF_0$\\
Internal rate of return (IRR) must satisfy: \\
$0 = CF_0 +\sum_i\dfrac{CF_i}{(1+IRR)^i}$ \\
Payback Interval:\\
$PI = \dfrac{PV}{-CF_0}$
