\section{Foundations of Modern Finance II}
\subsection*{Forward rates}
Forward interest rate between time $t-1$ and $t$:\\
$f_t=\dfrac{B_{t-1}}{B_t} - 1=\dfrac{(1+r_t)^t}{(1+r_{t-1})^{t-1}} - 1$\\
Expectation hypotesis (forward rates at time $0$ are predictors of future spot rates, which is not true):\\
$E_0[\tilde{r}_1(t)] = \dfrac{\left(1+r_{t+1}(0)\right)^{t+1}}{\left(1+r_t(0)\right)^t} - 1= f_{t+1}$
\subsection*{Forward pricing}
Current spot price: $S_0$ \\
Spot price at maturity (random): $\tilde{S}_T$\\
Forward price (fixed at time $0$): $F_T$ \\
Forward payoff is $\tilde{S}_T - F_T$\\
$PV_0(\tilde{S}_T) = e^{-yT}S_0$,\\ where $y$ is dividend yield \\
$F_T=e^{(r-y)T}S_0$, so dividend yield\\ $y = r- T\cdot ln\left(\dfrac{F_T}{S_0}\right)$\\
Currency forward price is:\\
$F_T=X_0\cdot e^{(r_{USD}-r_{CHF})\cdot T}$
\subsection*{Futures pricing}
Storage cost $Cost_t = c\cdot S_t$.\\
Net convenience yield $\hat{y} = y - c$, so\\
$H_T \approx F_T = e^{(r-\hat{y})T}S_0$\\
Backwardation in terms of convenience yield vs interest rate:
$\hat{y} - r = y - c -r > 0$\\
Contango: $H_T > S_0e^{rT}$\\
Backwardation: $H_T < S_0e^{rT}$
\subsection*{Interest rate swaps}
Fixed leg is paid at fixed rate $r_S$. \\
Floating leg at the end of each period $t$ is paid as spot risk-free rate $\tilde{r}_1(t-1)$\\
Forward rate to future spot rate:\\
$PV_0[\tilde{r}_1(T)\ at\ T + 1] = PV_0[f_{T+1}\ at\ T + 1]$\\
Present value of the fixed leg:
$r_S\times\sum_{u=1}^TB_u$\\
Present value of the floating leg of the swap: $\sum_{t=1}^TPV_0[\tilde{r}_1(t-1)\ at\ t]$\\
Swap rate:
$r_S = \dfrac{\sum_{t=1}^TB_t\cdot f_t}{\sum_{u=1}^TB_u} = \sum_{t=1}^Tw_t\times f_t$, where weights $w_t$ are:
$w_t=\dfrac{B_t}{\sum_{u=1}^TB_u}$\\
Alternative formula:
$r_s=\dfrac{1-B_T}{\sum_{u=1}^TB_U}$
\subsection*{Options}
$S$ underlying asset price (at time 0).\\
$S_T$ underlying asset price (at time T).\\
$B$ price of discount bond of par \$1 and maturity T ($B \leq 1$)\\
$K$ strike (excercise) price.\\
$T$ maturity (expriration) date.\\
$C$ price of call with strike $K$ and maturity $T$.\\
$P$ price of put with strike $K$ and maturity $T$.\\
European call option payoff:
$CF_T=max[0, S_T - K]$\\
The net payoff is: $max[S_T-K, 0] - C(1+r)^T$\\
Excercise value of a call is $S-K$.\\
Excercise value of a put is $K-S$.\\
Price bounds are: $max[S-KB,0] \leq C \leq S$\\
Put-Call parity: $C + BK = P + S$, where $B$ is $e^{-rT}$ if continious compounding is used.
\subsection*{Corporate securities as options}
Equity (E): A call option on firm's assets (A) with $K$ equal to its bond's redemption value.\\
Debt (D): A portfolio of firm's assets (A) and a short position in the call with $K$ equial to bond's face value $F$.\\
Warrant: Call option on firm's stock, with stock dilution as a result of excercise.\\
Convertible bond: A portfolio of straight bonds and a call on the firm's stock with $K$ related to the conversion ratio.\\
Callable bond: A portfolio combining straight bonds and a short position in a call on these bonds.\\
$A = D + E => D = A - E$\\
$E \equiv max[0, A - F]$\\
$D = A- E = A - max[0, A -F]$
\subsection*{Binomial pricing: single period}
$r$ is the interest rate\\
Stock price change:\\
\begin{tikzpicture}
  \matrix (tree) [matrix of nodes,column sep=1.5cm]
          {
            & $S_u=uS_0$ \\
            $S_0$ & \\
             & $S_d=dS_0$ \\
          };
          \draw[->] (tree-2-1)--(tree-1-2);
          \draw[->] (tree-2-1)--(tree-3-2);
\end{tikzpicture}

Riskless bond price change:\\
\begin{tikzpicture}
  \matrix (tree) [matrix of nodes,column sep=1.5cm]
          {
            & $B_0(1+r)$ \\
            $B_0$ & \\
             & $B_0(1+r)$ \\
          };
          \draw[->] (tree-2-1)--(tree-1-2);
          \draw[->] (tree-2-1)--(tree-3-2);
\end{tikzpicture}\\
Need to solve system of qquations:\\
$S_u\cdot a + (1+r)\cdot b = C_u \\ S_d\cdot a + (1+r)\cdot b = C_d$, \\
where $a$ amount of stock shares (option's delta), $b$ dollars invested into riskless bond $B$, $C_u$ is payoff in up state, $C_d$ is payoff in down state, so current market value of the call option is: $C_0 = S_0\cdot a + b$.
Alternative notation: \\
$\delta uS_0 + b(1+r) = C_u \\ \delta dS_0 + b(1+r) = C_d$\\
where unique solutions is:\\
$\delta = \dfrac{C_u-C_d}{(u -d)S_0},\ b=\dfrac{1}{1+r}\cdot\dfrac{uC_d - dC_u}{(u-d)}$, so\\
$C_0 = \delta S_0 + b = \dfrac{C_u-C_d}{u -d} + \dfrac{1}{1+r}\cdot\dfrac{uC_d - dC_u}{(u-d)}$\\
\subsection*{Risk-neutral probability}
$q_u = \dfrac{(1+r) -d}{u-d},\ q_d=\dfrac{u-(1+r)}{u-d}$\\
$C_0=\dfrac{q_uC_u + q_dC_d}{1+r} = \dfrac{E^Q[C_T]}{1+r}$,\\ where
$E^Q[\cdot]$ is expectation under risk-neutral probability $Q = (q, 1-q)$.
\subsection*{State prices}
$\phi_u=\dfrac{q}{1+r},\ \phi_d=\dfrac{1-q}{1+r}$
\subsection*{Black-Scholes-Merton formula}
$C_0=C(S_0,K,T,r,\sigma) = S_0N(x) - Ke^{-rT}N(x-\sigma\sqrt{T})$, where $x$ is: $x=\dfrac{ln\left(\dfrac{S_0}{Ke^{-rT}}\right)}{\sigma\sqrt{T}} + \dfrac{1}{2}\sigma\sqrt{T}$\\
So option delta ($\delta$) becomes $N(x) = \dfrac{\partial C}{\partial S}$.\\
$S_0\cdot N(x)$ is the dollar mount invested into stock.\\
$Ke^{-rT}N(x-\sigma\sqrt{T})$ is the dollar amount borrowed.
\subsection*{Put-Call parity with BSM formula}
$C + BK = P + S => P = C + e^{-rT}K - S$\\
$P = S\cdot N(x) - Ke^{-rT}N(x - \sigma\sqrt{T}) + e^{-rT}K-S$\\
$P = -S(1-N(x)) + Ke^{-rT}(1-N(x-\sigma\sqrt{T}))$
\subsection*{Option Greeks}
Delta: $\delta = \dfrac{\partial C}{\partial S}$\\
Omega: $\Omega = \dfrac{\partial C}{\partial S}\dfrac{S}{C}$\\
Gamma: $\Gamma = \dfrac{\partial \delta}{\partial S} = \dfrac{\partial^2C}{\partial S^2}$\\
Theta: $\Theta = \dfrac{\partial C}{\partial T}$\\
Vega: $v = \dfrac{\partial C}{\partial \sigma}$
